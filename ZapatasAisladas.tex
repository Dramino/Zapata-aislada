\documentclass[11pt,a4paper,fleqn]{article}
\usepackage[left=2.5cm, right=2.5cm, top=3cm, bottom=3cm]{geometry}
\usepackage[utf8]{inputenc}
\usepackage{ifthen}
\usepackage{amsmath} 
\usepackage[nomessages]{fp} %Para las operaciones matemáticas
\usepackage[spanish]{babel}	
\spanishdecimal{.}
\usepackage{amsmath,amsfonts,amsthm} % Math packages
\begin{document}
\newcommand\caZ{0.4}     % ancho de la columna
\newcommand\cbZ{0.6}     % largo de la columna
\newcommand\bZ{1.7}      % ancho de la zapata
\newcommand\lZ{3.5}      % largo de la zapata
\newcommand\dfZ{1.5}     % Profunidad del paño superior de la zapata
\newcommand\hZ{0.75}     % Peralte de la zapata
\newcommand\rZ{0.05}     % Recubrimiento
\newcommand\gammaS{1.6}  % Peso específico del suelo
\newcommand\fy{4200}     % Resistencia del acero
\newcommand\fc{200}      % Resistencia del concreto
\newcommand\qaS{20}      % Capacidad de carga del suelo
\newcommand\mxZ{15}      % Momento alrdedeor del eje X
\newcommand\myZ{6}       % Momento alrdedeor del eje Y
\newcommand\pZ{90}       % Carga axial
\newcommand\fcCarga{1.4} % Factor de carga
%Datos de armado
\newcommand\VarX{5}      % Número de varilla
\newcommand\distVarX{10} % Distancia 
\newcommand\VarY{5}      % Número de varilla
\newcommand\distVarY{10} % Distancia



%Peralte efectivo
\FPeval{\dZ}{round(\hZ-\rZ,2)}
%Compresión del concreto 
\FPeval{\fcc}{round(0.85*\fc,2)}

\section{Zapata aislada rectangular sujeta a flexo-compresión}
\subsection{Datos de carga}
A continuación se muestran las propiedades del suelo, la geometría de la zapata, las propiedades de los materiales y las cargas actuantes sin factorizar.
\begin{align*}
	&c_1=       \caZ    \text{\;m}                && \text{(Ancho de la columna)}\\
	&c_2=       \cbZ    \text{\;m}                && \text{(Largo de la columna)}\\
	&B=         \bZ     \text{\;m}                && \text{(Ancho de la zapata)}\\
	&L=         \lZ     \text{\;m}                && \text{(Largo de la zapata)}\\
	&d_f=       \dfZ    \text{\;m}                && \text{(Profunidad de la zapata)}\\
	&h=         \hZ     \text{\;cm}               && \text{(Peralte de la zapata)}\\
	&d=         \dZ     \text{\;cm}               && \text{(Peralte efectivo de la zapata)}\\
	&r=         \rZ     \text{\;cm}               && \text{(Recubrimiento de la zapata)}\\
	&\gamma_s=  \gammaS \text{\;ton/m$^3$}        && \text{(Peso espefícico del suelo)}\\
	&f_y=       \fy     \text{\;kg/cm$^2$}        && \text{(Resistencia del acero)}\\
	&f^{'}_c=   \fc     \text{\;kg/cm$^2$}        && \text{(Resistencia del concreto)}\\
	&f^{''}_c=  0.85(\fc)=\fcc\text{\;kg/cm$^2$}  && \text{(Resistencia del concreto)}\\
	&q_a=       \qaS    \text{\;ton/m$^2$}        && \text{(Capacidad de carga del suelo)}\\
	&P=         \pZ     \text{\;ton}              && \text{(Fuerza axial actuante)} \\
	&M_{x}=     \mxZ    \text{\;ton-m}            && \text{(Momento alrededor del eje x)}\\
	&M_{y}=     \myZ    \text{\;ton-m}            && \text{(Momento alrededor del eje y)}	
\end{align*}

\subsection{Peso de la zapata}
Para el cálculo de las presiones de la zapata se requiere obtener las presiones adicionales debido al peso propio de la zapata y del suelo
%Peso del suelo
\FPeval{\ppS}{round((\lZ*\bZ-(\caZ*\cbZ))*\dfZ*\gammaS,2)}
%Peso de la zapata
\FPeval{\ppZ}{round(\lZ*\bZ*\hZ*2.4,2)}
%Peso del dado
\FPeval{\ppD}{round(\caZ*\cbZ*\dfZ*2.4,2)}
%Peso total
\FPeval{\ppC}{round(\ppS+\ppZ+\ppD,2)}
%Peso total
\FPeval{\pU}{round(\pZ+\ppC,2)}

\begin{align*}
	Pp_{s}&=\left[LB-(c_1c_2)\right]d_f\gamma_s&=&\left[(\lZ)(\bZ)-(\caZ)(\cbZ)\right](\dfZ)(\gammaS)&=&\;\ppS \text{\;ton}\\
	Pp_{z}&=L(B)(h)(2.4)&=&(\lZ)(\bZ)(\hZ)(2.4)&=&\;\ppZ \text{\;ton}\\
	Pp_{d}&=c_1(c_2)(d_f)(2.4)&=&(\caZ)(\cbZ)(\dfZ)(2.4)&=&\;\ppD \text{\;ton}\\
	Pp_{c}&=Pp_{s}+Pp_{z}+Pp_{d}&=&\ppS+\ppZ+\ppD&=&\;\ppC \text{\;ton}
\end{align*}

donde $Pp_{s}$ es el peso del suelo, $Pp_{z}$ es el peso de la zapata, $Pp_{d}$ es el peso del dado y $Pp_{c}$ es el peso de la cimentación. 
Por lo que el peso total actuante es:

\begin{align*}
	P_{U}=P+Pp_{c}=\pZ+\ppC=\pU \text{ ton}
\end{align*}

\subsection{Presión actuante}
Una vez obtenido la carga axial actuante debajo de la zapata aislada se obtienen las presiones máximas y mínimas como se muestra a continuación

%Presión máxima y míminma
\FPeval{\qMax}{round(\pU/(\bZ*\lZ)+6*\mxZ/(\bZ*pow(2,\lZ))+6*\myZ/(\lZ*pow(2,\bZ)),2)}
\FPeval{\qMin}{round(\pU/(\bZ*\lZ)-6*\mxZ/(\bZ*pow(2,\lZ))-6*\myZ/(\lZ*pow(2,\bZ)),2)}

\begin{align*}
	q_{max}=\frac{P_{U}}{BL}+\frac{6M_x}{BL^2}+\frac{6M_y}{LB^2}=\frac{\pU}{(\bZ)(\lZ)}+\frac{6(\mxZ)}{\bZ(\lZ)^2}+\frac{6(\myZ)}{\lZ(\bZ)^2}=\qMax \text{ t/m$^2$}\\
	q_{min}=\frac{P_{U}}{BL}-\frac{6M_x}{BL^2}-\frac{6M_y}{LB^2}=\frac{\pU}{(\bZ)(\lZ)}-\frac{6(\mxZ)}{\bZ(\lZ)^2}-\frac{6(\myZ)}{\lZ(\bZ)^2}=\qMin \text{ t/m$^2$}
\end{align*}

Sin embargo la presión de diseño se obtiene considerando que la presión es uniforme en una sección de la zapata, para ello primero se obtienen las excentricidades

%Excentricidades
\FPeval{\eX}{round(\myZ/\pU,2)}
\FPeval{\eY}{round(\mxZ/\pU,2)}

\begin{align*}
	e_x=\frac{M_y}{P_U}=\frac{\myZ}{\pU}=\eX \text{ m}\\
	e_y=\frac{M_x}{P_U}=\frac{\mxZ}{\pU}=\eY \text{ m}
\end{align*}

Esta excentricidad reduce las secciones de la zapata de tal manera que

%Area reducida
\FPeval{\bRedZ}{round(\bZ-2*\eX,2)}
\FPeval{\lRedZ}{round(\lZ-2*\eY,2)}

\begin{align*}
	B'=B-2e_x=\bZ-2(\eX)=\bRedZ \text{ m}\\
	L'=L-2e_y=\lZ-2(\eY)=\lRedZ \text{ m}
\end{align*}

por lo que la presión de diseño es
%presión de diseño
\FPeval{\qD}{round(\fcCarga*\pU/(\bRedZ*\lRedZ),2)}

\begin{align*}
	q_D=\frac{FC(P_U)}{B'L'}=\frac{\fcCarga(\pU)}{(\bRedZ)(\lRedZ)}=\qD \text{ t/m$^2$}
\end{align*}

\section{Revisión por penetración}
Para la revisión a penetración se obtiene el área crítica

%Área crítica
\FPeval{\aCrit}{round(2*\dZ*(\caZ+\cbZ+2*\dZ),2)}

\begin{align*}
	A_{cr}=2d(c_1+c_2+2d)=2(\dZ)(\caZ+\cbZ+2(\dZ))=\aCrit \text{ m$^2$}
\end{align*}

En esta zona crítica existe una carga en contra de la carga axial . Dicha fuerza es

%Fuerza a favor
\FPeval{\pFavor}{round(\qD/\fcCarga*(\caZ+\dZ)*(\cbZ+\dZ),2)}
\begin{align*}
	P'=\frac{q_d}{FC}(c_1+d)(c_2+d)=\frac{\qD}{\fcCarga}(\caZ+\dZ)(\cbZ+\dZ)=\pFavor \text{ ton}
\end{align*}

Por lo que el cortante actuante es

%Cortante por penetración
\FPeval{\vuPen}{round((\pZ+\ppD-\pFavor)*\fcCarga,2)}
\begin{align*}
	V=FC(P+P_d-P')=\fcCarga(\pZ+\ppD-\pFavor)=\vuPen \text{ ton}
\end{align*}



Para revisar si existe transmisión de momento se evalúa que $Mux$ no exceda de $0.2Vu(d)$ por lo que

%Revisión de transmisión de momento en My
\FPeval{\vuMin}{round(0.2*\vuPen*\dZ,2)}
 \FPifgt \mxZ \vuMin  \FPeval{\pruebaVer}{1} \else \FPeval{\pruebaVer}{2} \fi
Revisión si esto pasa, se pondra el valor de la prueba que es \pruebaVer
\ifthenelse{\pruebaVer=1}  
	{
	%Hay transmisión de momento
	\begin{align*}
		0.2Vu(d)=0.2(\vuPen)(\dZ)=\vuMin<\mxZ 
	\end{align*}

entonces si hay transmisión de momento, por lo que se requiere obtener 
	%Jc dirección x
	\FPeval{\jcX}{round(\dZ*pow(3,\caZ+\dZ)/6+(\caZ+\dZ)*pow(3,\dZ)/6+\dZ*(\cbZ+\dZ)*pow(2,(\caZ+\dZ))/2,2)}
	%alfa dirección x
	\FPeval{\alphaX}{round(1-1/(1+0.67*pow(0.5,(\caZ+\dZ)/(\cbZ+\dZ))),2)}
	\begin{align*}
		&J_c&=&\frac{d(c_1+d)^3}{6}+\frac{(c_1+d)(d)^3}{6}+\frac{d(c_2+d)(c_1+d)^2}{2}\\
		&J_c&=&\frac{\dZ(\caZ+\dZ)^3}{6}+\frac{(\caZ+\dZ)(\dZ)^3}{6}+\frac{\dZ(\cbZ+\dZ)(\caZ+\dZ)^2}{2}=\jcX\\
		&\alpha&=&1-\frac{1}{1+0.67\sqrt{(c_1+d)/(c_2+d)}}\\
		&\alpha&=&1-\frac{1}{1+0.67\sqrt{(\caZ+\dZ)/(\cbZ+\dZ)}}=\alphaX
	\end{align*}
	
	Con estos valores se obtiene el cortante máximo siendo
	%Cortante máximo
	\FPeval{\vuMaXPen}{round(\vuPen/\aCrit+\alphaX*\mxZ*\bZ/\jcX,2)}
	
	\begin{align*}
		V_{max}=\frac{V}{A_{cr}}+\frac{\alpha M_x B}{J_c}=\frac{\vuPen}{\aCrit}+\frac{(\alphaX)(\mxZ)(\bZ)}{\jcX}=\vuMaXPen \text{ ton}
	\end{align*}
	} 
%\else
	{
	%No hay transmisión de momento
	\begin{align*}
		0.2Vu(d)=0.2(\vuPen)(\dZ)=\vuMin>\mxZ 
	\end{align*}
	Por lo que no hay transmisión de momentos, dando el cortante máximo como
	
	\FPeval{\vuMaXPen}{round(\vuPen/\aCrit,2)}
	
	\begin{align*}
		V_{max}=\frac{V}{A_{cr}}=\frac{\vuPen}{\aCrit}=\vuMaXPen \text{ ton}
	\end{align*}
  }
%\fi

En la otra dirección se tiene

%Revisión de transmisión de momento en My
\FPifgt \myZ \vuMin  \FPeval{\pruebaVer}{1} \else \FPeval{\pruebaVer}{2} \fi
%\FPifgt \myZ \vuMin  
\ifthenelse{\pruebaVer=1}
	{
	%Hay transmisión de momento
	\begin{align*}
		0.2Vu(d)=0.2(\vuPen)(\dZ)=\vuMin<\myZ 
	\end{align*}

entonces si hay transmisión de momento, por lo que se requiere obtener 
	%Jc dirección x
	\FPeval{\jcY}{round(\dZ*pow(3,\cbZ+\dZ)/6+(\cbZ+\dZ)*pow(3,\dZ)/6+\dZ*(\caZ+\dZ)*pow(2,(\cbZ+\dZ))/2,2)}
	%alfa dirección x
	\FPeval{\alphaY}{round(1-1/(1+0.67*pow(0.5,(\cbZ+\dZ)/(\caZ+\dZ))),2)}
	\begin{align*}
		&J_c&=&\frac{d(c_2+d)^3}{6}+\frac{(c_2+d)(d)^3}{6}+\frac{d(c_1+d)(c_2+d)^2}{2}\\
		&J_c&=&\frac{\dZ(\cbZ+\dZ)^3}{6}+\frac{(\cbZ+\dZ)(\dZ)^3}{6}+\frac{\dZ(\caZ+\dZ)(\cbZ+\dZ)^2}{2}=\jcY\\
		&\alpha&=&1-\frac{1}{1+0.67\sqrt{(c_2+d)/(c_1+d)}}\\
		&\alpha&=&1-\frac{1}{1+0.67\sqrt{(\cbZ+\dZ)/(\caZ+\dZ)}}=\alphaY
	\end{align*}
	
	Con estos valores se obtiene el cortante máximo siendo
	%Cortante máximo
	\FPeval{\vuMaXPenY}{round(\vuPen/\aCrit+\alphaY*\myZ*\lZ/\jcY,2)}
	
	\begin{align*}
		V_{max}=\frac{V}{A_{cr}}+\frac{\alpha M_y L}{J_c}=\frac{\vuPen}{\aCrit}+\frac{(\alphaY)(\myZ)(\lZ)}{\jcY}=\vuMaXPenY \text{ ton}
	\end{align*}
	} 
%\else
	{
	%No hay transmisión de momento
	\begin{align*}
		0.2Vu(d)=0.2(\vuPen)(\dZ)=\vuMin>\myZ 
	\end{align*}
	Por lo que no hay transmisión de momentos, dando el cortante máximo como
	
	\FPeval{\vuMaXPenY}{round(\vuPen/\aCrit,2)}
	
	\begin{align*}
		V_{max}=\frac{V}{A_{cr}}=\frac{\vuPen}{\aCrit}=\vuMaXPenY \text{ ton}
	\end{align*}
  }
%\fi


\FPeval{\vuMaXPenXY}{max(\vuMaXPenY,\vuMaXPen)}
Una vez obtenidos los cortantes máximos por penetración en ambas direcciones se toma el mayor el cual es

\begin{align*}
		V_{max}=\vuMaXPenXY \text{ ton/m$^2$}
	\end{align*}

Con base el Artículo 5.3.6.4 ecuación (5.3.36) el cortante máximo a penetración es:

\FPeval{\vcrPen}{round(0.65*(\fc)^0.5*10,2)}


\begin{align*}
		V_{cr}=F_R\sqrt{f^{'}_{c}}=0.65\sqrt{\fc(100)}=\vcrPen \text{ ton/m$^2$}
\end{align*}
	
\subsection{Revisión como viga ancha}
\subsubsection{Momento}
Para la revisión a momento, se tiene que los limites superiores e inferiores de cuantías son:

%Cuantías máximas y mínimas
\FPeval{\pmin}{round((0.7*root(2,\fc))/\fy,4)}
\FPeval{\pmax}{round(\fcc/\fy*(6000*0.85)/(\fy+6000),4)}

\begin{align*}
	&\rho_{min}=\frac{0.7\sqrt{fc'}}{fy}                   =\frac{0.7\sqrt{\fc}}{\fy}                     =\pmin\\
	&\rho_{max}=\frac{fc''}{fy}\frac{6000\beta}{fy+6000}   =\frac{\fcc}{\fy}\frac{6000(0.85)}{\fy+6000}   =\pmax
\end {align*}

Considerando que la cuantía es

\begin{align*}
	&\rho=\frac{A_s}{(B)(h)}
\end {align*}

donde $A_s$ es el área de acero, se tiene que las áreas máximas y mínimas considerando un metro de ancho son:

%Áreas de acero máximas y mínimas
\FPeval{\dZcm}{round(100*\dZ,2)}
\FPeval{\asMin}{round(\pmin*100*\dZcm,2)}
\FPeval{\asMax}{round(\pmax*100*\dZcm,2)}
\begin{align*}
	&A_{s_{min}}=\rho_{min}(100)(d)=\pmin(100)(\dZcm)= \asMin \text{ cm$^2$/m} \\
	&A_{s_{max}}=\rho_{max}(100)(d)=\pmax(100)(\dZcm)= \asMax \text{ cm$^2$/m}
\end{align*}

De esto se puede deducir que el momento máximo permitido para la cuantía máxima de acero es:
\FPeval{\mMaxPer}{round(0.9*\fcc*100*pow(2,\dZcm)/200000,2)}

\begin{align*}
	&M_{max}=\frac{0.9f^{''}_c b d^2}{2\times 10^5}=\frac{0.9(\fcc)( 100)( \dZcm)^2}{2\times 10^5}= \mMaxPer \text{ ton-m}
\end{align*}

Dado que es una fuerza repartida, el momento máximo se obtiene del paño del dado a la zapata por loq ue las distancias $x$ y $y$ son:
\FPeval{\dxM}{round((\bZ-\caZ)/2,2)}
\FPeval{\dyM}{round((\lZ-\cbZ)/2,2)}

\begin{align*}
	&x=\frac{B-c_1}{2}=\frac{(\bZ-\caZ)}{2}= \dxM \text{ m} \\
	&y=\frac{L-c_2}{2}=\frac{(\lZ-\cbZ)}{2}= \dyM \text{ m} 
\end{align*}

Por lo que los momentos máximos son
%Momentos
\FPeval{\mMaxX}{round(\qD*\dxM^2/2,2)}
\FPeval{\mMaxY}{round(\qD*\dyM^2/2,2)}

\begin{align*}
	&M_x=\frac{q_d x^2}{2}=\frac{(\qD)\dxM^2}{2}= \mMaxX \text{ ton-m} \\
	&M_y=\frac{q_d y^2}{2}=\frac{(\qD)\dyM^2}{2}= \mMaxY \text{ ton-m} 
\end{align*}

El momento máximo en ambas direcciones es menor al límite por lo que se procede a obtener la cantidad de acero.

%Acero
\FPeval{\asTrX}{round((1-root(2,(1-2*100000*\mMaxX/(0.9*100*pow(2,\dZcm)*\fcc))))/(\fy/(100*\dZcm*\fcc)),2)}
\FPeval{\asTrY}{round((1-root(2,(1-2*100000*\mMaxY/(0.9*100*pow(2,\dZcm)*\fcc))))/(\fy/(100*\dZcm*\fcc)),2)}

\begin{align*}
	&A_s=\frac{1-\sqrt{1-\frac{2M_r}{F_R(b)(d^2)(fc'')}}}{\frac{fy}{b(d)(fc'')}}\\
	&A_{sx}=\frac{1-\sqrt{1-\frac{2(\mMaxX\times 10^5)}{0.9(100)(\dZcm^2)(\fcc)}}}{\frac{\fy}{100(\dZcm)(\fcc)}}=\asTrX \text{ cm$^2$}\\
	&A_{sy}=\frac{1-\sqrt{1-\frac{2(\mMaxY\times 10^5)}{0.9(100)(\dZcm^2)(\fcc)}}}{\frac{\fy}{100(\dZcm)(\fcc)}}=\asTrY \text{ cm$^2$}
\end {align*}

por lo que se colocara un armado con varilla del \# \VarX \; a cada \distVarX \;cm en la dirección x y varilla del \# \VarY \;a cada \distVarY \;cm en la dirección y

\subsubsection{Cortante}
Para ser considerado como viga ancha se requiere que la realción largo espesor sea mayor a 4 por lo que 
%Areas de varilla
%\FPeval{\asVarX}{round(0.07917304\VarX^2,2)}
%\FPeval{\asVarY}{round(0.07917304\VarX^2,2)}
\end{document}